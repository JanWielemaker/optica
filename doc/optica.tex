\documentclass[11pt,twocolumn]{article}
\usepackage[dutch]{babel}
\selectlanguage{dutch}
\usepackage{html}
\usepackage{pl}
\usepackage{psfig}
\onefile
\renewcommand{\runningtitle}{Optica}
\newcommand{\tick}[1]{\item #1\hfill\\}

\setcounter{topnumber}{100}
\renewcommand{\topfraction}{0.9}
\setcounter{totalnumber}{100}
\renewcommand{\bottomfraction}{0.9}
\renewcommand{\textfraction}{0.1}
\renewcommand{\dbltopfraction}{0.9}
\renewcommand{\dblfloatpagefraction}{0.8}
\renewcommand{\floatpagefraction}{0.8}

\makeatletter
\def\fps@figure{h}
\makeatother

\begin{document}

\title{\vskip -1.2in
       Optica --- Simulatie en leeromgeving}
\author{Jan Wielemaker \\
        SWI, Universiteit van Amsterdam \\
        E-mail: jan@swi.psy.uva.nl}

\maketitle

\section*{Wat is het SWI}

De vakgroep {\em Sociaal Wetenschappelijke Informatica} van de
Universiteit van Amsterdam werkt ondermeer aan kennisintensieve
(computer-)systemen en intelligente onderwijssystemen. Om dit onderzoek
te ondersteunen is XPCE/Prolog ontwikkeld, een software
ontwikkelomgeving die, middels de integratie van een geavanceerde
grafische omgeving en de AI taal Prolog, zeer geschikt is voor
interactieve kennisintensieve software.

De optische simulatieomgeving {\em Optica} is ontwikkeld voor het SGW%
	\footnote{Stichting Gedrags Wetenschappen}
project ``Inductief leren'', bedoeld om de invloed van algemene
leervaardigheden (b.v.\ systematisch werken) op het leren met een
simulatieomgeving te onderzoeken.


\section*{Wat is optica?}

Optica biedt een simulatieomgeving met lenzen, diverse lichtbronnen,
schermen en meetinstrumenten. 

Optica kan op twee manieren gebruikt worden. Gevorderden kunnen met de
gehele verzameling instrumenten naar eigen inzicht experimenteren (een
natuurkundige die betrokken is bij het experiment waarvoor optica
ontwikkeld is heeft er een zoomlens in gemaakt).

Het is echter ook mogelijk om experimenteeromgevingen in optica te
configureren. Een experimenteeromgevingen bestaat uit een
deelverzameling van de instrumenten, waarbij ook de
manipulatiemogelijkheden ingeperkt kunnen worden. Aan een
experimenteeromgevingen kan een tekst met theorie, een handleiding en
een toets gekoppeld worden.


\section*{Instrumenten}

Optica bevat de hieronder genoemde instrumenten.  Er is geen beperking
aan het aantal instrumenten wat gelijktijdig gebruikt kan worden.

\begin{itemize}
    \setlength{\itemsep}{0pt}
    \tick{lens}
Met instelbare diameter, bolling (van plat tot cirkel, inclusief
hol-bolle lenzen), kleinste dikte en brekingsindex materiaal.  Lenzen
kunnen gemodeleerd worden met de wet van Snellius (inclusief
reflectie) of met de ideale lenzenwet.

    \tick{laserstraal}
Kan overal opgesteld worden.  Hoek met de optische as is instelbaar.

    \tick{punt lichtbron}
Kan overal opgesteld worden.

    \tick{Grote diffuse lamp}
Voor beschijning optisch scherm.

    \item{Scherm met gaatjes in de vorm van een `L'}
    \tick{Scherm voor beeldvorming}
De twee schermen kunnen gezamelijk gebruikt worden om beeldvorming en
vergroting te illustreren.  
    \item{`Oog' om langs de optische as naar de lichtbron te `kijken'}
    \item{Meetinstrumenten voor afstanden en hoeken}
    \item{`Virtuele' verlenging voor lichtstralen}
    \item{Simpele calculator}
    \item{Grafieken en tabellen}
    \item{Aantekenblok}
\end{itemize}


\section*{Computer configuratie}

Optica is beschikbaar voor Windows 95, Windows NT en Unix. Voor soepele
werking is een machine met Pentium 100 en 16 MB geheugen gewenst. De
Optica distributie past op een enkele 3.5'' floppy disk. Het systeem
vraagt ca. 3.5 MB ruimte op de harde schijf.


\section*{Voorbeelden}

Hieronder volgen enkele schermen die de mogelijkheden van optica
illustreren.

\postscriptfig[width=\linewidth]{fig1}{Lens met laser, afstandmeter en
hoekmeter}

\postscriptfig[width=\linewidth]{fig2}{Beeldvorming met een lens,
ge\"illustreerd met scherm en puntbron}

\postscriptfig[width=\linewidth]{fig3}{Brandpuntafstand van een
negatieve lens, gemeten d.m.v.\ virtuele verlenging stralen en afstandmeter}

\postscriptfig[width=\linewidth]{fig4}{Configuratiescherm
voor een experimenteeromgeving.	 Er wordt een omgeving gemaakt voor
illustratie van de brekingswetten van Snellius, bestaande uit een
halve glazen cirkel, een laserstraal en diverse meet- en
manipulatieexperimenten.}

\end{document}
